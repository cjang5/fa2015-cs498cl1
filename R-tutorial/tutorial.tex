\documentclass[12pt]{article}

\usepackage[utf8]{inputenc}


%%% PAGE DIMENSIONS
\usepackage{geometry} % to change the page dimensions
\geometry{a4paper} % or letterpaper (US) or a5paper or....

\usepackage{hyperref}
\hypersetup{}

\title{\huge Tutorial 1: R Basics}
\author{CS 498 CL1: Probability in CS}
\date{Updated September 2nd, 2015}

\begin{document}
\maketitle

\section{First Things First}

This tutorial is to familiarize you with R. It goes over basic functions,
operator overloading and selecting and reading data. Before you start,
make sure you have correctly installed R/R studio. As you go through this tutorial,
test out snippets of code yourself so that you can better understand how things
work in R.

Also, feel free to use the \texttt{help(function)} or \texttt{?function} commands
at any point to learn more about what a function does. Needless to say, google
is your friend.

\section{Basics}

\begin{itemize}

\item \emph{The \texttt{c} command}

The \texttt{c} command is the easiest way to input a bunch of values to a vector.
In order to assign a variable name, use "\texttt{<-}". For instance:

\begin{verbatim}
> wow <- c(5, 234, 23, 566, 56)
> wow
[1] 5 234 23 566 56
> name <- "Doge"
> name
[1] "Doge"
\end{verbatim}

% There's so little difference between "-" and "<-" that it's not worth
% mentioning.
%Using "\texttt{<-}" declares a variable within the user workspace, whereas using
%"$=$" does not. To learn more about these differences, look
%\href{http://stackoverflow.com/questions/1741820/assignment-operators-in-r-and}{here}
%(http://stackoverflow.com/questions/1741820/assignment-operators-in-r-and).

\item \emph{Support: The \texttt{help} and \texttt{example} Functions}

Call the help function anytime you need more info about certain function arguments,
objects etc. In R, this usually opens a browser window, and in R studio,
information is displayed within the application. The syntax is \texttt{help(function)}
and \texttt{example("function")}.

\begin{verbatim}

> help(hist)
# This should open a browser window.
> example("hist")
hist> op <- par(mfrow = c(2, 2))

hist> hist(islands)
Hit <Return> to see next plot:
# Check out what the function does by hitting <Return>
\end{verbatim}
You can try out examples for other functions such as \texttt{boxplot}, \texttt{read.csv}, etc.

\item \texttt{n:m}

Create a vector containing values within a range:

\begin{verbatim}
> amaze <- 24:30
> amaze
[1] 24 25 26 27 28 29 30
\end{verbatim}

\item Other useful basic commands worth trying out.
\begin{verbatim}
# These help you maneuver to the R directory you are working in.
# This is important in case you want to load in local csv files.
# In R studio, you could set your directory just by clicking a few
# buttons; your workflow might be different depending on your OS.

getwd() # Gives you the current working directory
setwd() # Change working directory
        # Useful when you need to set your directory where your csv files
        # are located. Again, R studio has a click interface for this.
dim() # Returns the dimension of the variable
\end{verbatim}
\end{itemize}

% I removed the definition of typeof(). In my experience of using
% R, because the type system is so jank, most of the time this function
% doesn't give me any informative information.

\section{Data Structures}
Data structures are usually of the following types:
\begin{itemize}

\item{\emph{Vectors:}} \emph{Sequence of data elements of a single type.}

\begin{verbatim}
> my.strings <- c("Hello", "world")              # Vector of length two
> my.strings
[1] "hello" "world"

# Now want to play with vectors of numbers
> a <- c(1, 2, 3, 4, 5, 6, 7, 8, 9, 10)          # Equal to `a <- 1:10`
> b <- c(11, 12, 13, 14, 15, 16, 17, 18, 19, 20) # Equal to `b <- 11:20`
> a+b # Addition, multiplication etc. if vectors of same dimensions
[1] 12 14 16 18 20 22 24 26 28 30

> a*b # Multiplies individual elements, NOT a dot product
[1]  11  24  39  56  75  96 119 144 171 200

# Scalar addition; Logical operators compare each element of a vector
> a + 10 == b
[1] TRUE TRUE TRUE TRUE TRUE TRUE TRUE TRUE TRUE TRUE
# NOTE: A vector (or data frame) of booleans is called a mask.
# We won't use them profusely this semester, but they are good to
# know about if you wish to continue learning R.

# To index elements within the vector
> b[4]
[1] 14
# Note that R indexes from 1, rather than 0.

# You can alternatively access a vector with a mask
> mask <- c(T, F, F, F, F, F, F, F, F, T)  # T and F are short
                                           # for TRUE and FALSE
> masked.out <- a[mask]
> masked.out
[1] 1 10

\end{verbatim}

\item{\emph{Data frames:}} \emph{List of vectors of equal lenghts; used for storing data tables.}
\begin{verbatim}
# Create a Data frame for students, identified by
# registration numbers (regnum); with indicated
# marks and a grade for conduct

> regnum <- 1:10
> marks <-c(12, 45, 67, 54, 34, 68, 88, 33, 22, 25)  # This is of length 10
> conduct <-c("A", "B", "A", "A", "C", "B", "B", "A", "B", "A")

> df <- data.frame(regnum, marks, conduct)
   regnum marks conduct
1       1    12       A
2       2    45       B
3       3    67       A
4       4    54       A
5       5    34       C
6       6    68       B
7       7    88       B
8       8    33       A
9       9    22       B
10     10    25       A

# There are a couple of ways you could access the columns in this
# data frame. We show one here, to not overwhelm you guys.

# The "$" sign notation in R (which you might see now and then) means
# we're accessing an attribute of an object. In a data frame, each
# column counts as an attribute.
> df$conduct
 [1] A B A A C B B A B A
Levels: A B C  # If a non-numeric variable takes on discrete values,
               # these discrete values are called 'Levels'

# Extract data from the data frame using subset()

> subset(df, regnum == 5)
5      5    34       C

# Extract data corresponding to students with conduct grade "A"
> subset(df, conduct == "A")
   regnum marks conduct
1       1    12       A
3       3    67       A
4       4    54       A
8       8    33       A
10     10    25       A

\end{verbatim}

\section{Reading in Data}

You can read in data with commands such as \texttt{read.csv()} (Comma delimited files),
\texttt{read.table} (Usually works for text files and tables), \texttt{read.xls}
(Excel Files).

\begin{verbatim}

# First, set the correct working directory
> getwd() # get the current working directory
> setwd("<your working folder path here>")

> mydata<- read.csv("filename.csv", header=TRUE, sep=",",)

> head(mydata)
> tail(mydata)
# use these to check whether your data has been imported correctly
\end{verbatim}


\section{More Resources to Get Started}

\item R tutor - \href{http://www.r-tutor.com/taxonomy/term/220/0}{http://www.r-tutor.com/taxonomy/term/220/0}

\item R for Beginners - Emmanual Paradis:

\href{https://cran.r-project.org/doc/contrib/Paradis-rdebuts\_en.pdf}{https://cran.r-project.org/doc/contrib/Paradis-rdebuts\_en.pdf}

\item Codeschool - \href{http://tryr.codeschool.com/}{http://tryr.codeschool.com/}



\end{itemize}

\end{document}
